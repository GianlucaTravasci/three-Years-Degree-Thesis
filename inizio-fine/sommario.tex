% !TEX encoding = UTF-8
% !TEX TS-program = pdflatex
% !TEX root = ../tesi.tex

%**************************************************************
% Sommario
%**************************************************************
\cleardoublepage
\phantomsection
\pdfbookmark{Sommario}{Sommario}
\begingroup
\let\clearpage\relax
\let\cleardoublepage\relax
\let\cleardoublepage\relax

\chapter*{Sommario}

Il presente documento descrive il lavoro svolto durante il periodo di stage, della durata di 320 ore, dal laureando Gianluca Travasci presso l'azienda \myAzienda  di Pordenone (PN).
\\
\\
Lo scopo dello stage è stato quello di implementare al software F12, realizzato da \myAzienda, la possibilità di gestire, archiviare e configurare dei protocolli. Il suddetto lavoro è stato svolto per far fronte alla necessità di una cooperativa della zona di Pordenone di gestire in modo efficiente ed immediato ogni aspetto della protocollazione.
\\
\\
Gli obiettivi da raggiungere erano molteplici: in primo luogo era richiesto lo studio della proposta d'appalto per poi realizzare una breve analisi dei requisiti corredata da mockup da presentare all'azienda proponente.
In secondo luogo era richiesto lo studio del software F12 e del framework ad esso collegato per poi cominciare a lavorarci.\\
Infine era richiesta la realizzazione dell'infrastruttura base lato server, la realizzazione delle principali maschere lato client e l'interazione con il database mediante linguaggio SQL.

%\vfill
%
%\selectlanguage{english}
%\pdfbookmark{Abstract}{Abstract}
%\chapter*{Abstract}
%
%\selectlanguage{italian}

\endgroup			

\vfill

