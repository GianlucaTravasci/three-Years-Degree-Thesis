% !TEX encoding = UTF-8
% !TEX TS-program = pdflatex
% !TEX root = ../tesi.tex

%**************************************************************
\chapter{Conclusioni}
\label{cap:conclusioni}
%**************************************************************
Quest'ultimo capitolo contiene un'analisi riassuntiva degli obiettivi raggiunti, delle conoscenze acquisite e le conclusioni sull'attività svolta.
%**************************************************************
\section{Obiettivi raggiunti}
Gli obiettivi prefissati nel piano di lavoro sono stati completamente raggiunti. Ogni giorno, in seguito all'analisi degli obiettivi raggiunti e delle problematiche da risolvere, veniva redatto un documento informale contenente a grandi linee le soluzioni che si intendevano intraprendere, concordatamene con il tutor aziendale.
\\
A livello tecnologico ho acquisito una buona padronanza dei linguaggi di sviluppo web quali PHP e JavaScript che per la mia futura carriera sono fondamentali. 

%**************************************************************
\section{Conoscenze acquisite}
Oltre a quanto detto sopra, la cosa più importante che ho imparato in questo stage è stata la capacità di realizzare codice il più possibile mantenibile e sicuro dal punto di vista delle vulnerabilità.
\\ 
Quest'ultimo punto è quello a mio avviso più importante e formativo per uno studente che si affaccia al mondo del lavoro e che come me vorrebbe affacciarsi al settore della sicurezza informatica.
%**************************************************************
\section{Conclusioni}
Personalmente ritengo questa esperienza di stage molto positiva e formativa, sotto
molti punti di vista.
\\
Innanzitutto mi ha permesso di affacciarmi al mondo del lavoro nell'ambito informatico per la prima volta: entrare in contatto con un team di professionisti del settore è un esperienza molto formativa che sicuramente non può essere insegnata in un contesto universitario. 
\\
Penso che un opportunità di stage come quella offerta dal nostro percorso di studi, di cui sono completamente soddisfatto, è fondamentale per mettere in pratica, e quindi comprendere ancora meglio, tutte le conoscenze teoriche apprese durante questi anni di studio.
\\
Sicuramente le nozioni insegnate dal corso di Laurea in Informatica, ma soprattutto il metodo di studio che viene comunicato, mi hanno permesso di imparare nuove tecnologie senza difficoltà e di affrontare nuove sfide senza paura. Questo percorso mi ha permesso anche di comprendere le regole aziendali, dal rispettare gli orari lavorativi fino alla pianificazione e al raggiungimento degli obiettivi, molto diverse dall'ambito universitario.
\\
Mi ritengo molto soddisfatto del lavoro svolto in quanto il prodotto sarà venduto all'azienda proponente e, da come mi è stato riferito dal tutor aziendale, sarà ripreso per adattarlo all'esigenza di realizzare un gestionale per la fatturazione elettronica.
\\
Concludo dicendo che questo percorso mi ha permesso di conoscere meglio me stesso, i
miei limiti, ma anche i miei punti di forza e le aspirazioni.
%**************************************************************
