% !TEX encoding = UTF-8
% !TEX TS-program = pdflatex
% !TEX root = ../tesi.tex

%**************************************************************

\chapter{Verifica e validazione}
\label{cap:verifica-validazione}
%**************************************************************

\section{Verifica}

\subsection{Analisi statica}
L'analisi statica fa uso di tecniche che non richiedono l'esecuzione del prodotto software, mirando ad avere un indice di qualità del codice tramite la lettura.
\\
L'analisi statica è stata applicata per tutta la durata del periodo di codifica, nei seguenti modi:
\begin{itemize}
    \item rileggendo innanzitutto attentamente il codice prodotto durante tutta la fase di codifica, facendo uso di tecniche come \emph{walkthrough}\glsfirstoccur e \emph{inspection}\glsfirstoccur;
    
    \item tenendo sotto controllo la qualità del codice grazie a Psalm.
\end{itemize}

\subsection{Analisi dinamica}
L'analisi dinamica richiede l'esecuzione del prodotto software. Si avvale tipicamente di test progettati per essere utilizzabili nel momento in cui si effettua una modifica al software.
\\
Allo scopo di effettuare l'analisi dinamica, sono stati stesi ed effettuati dei test di unità sulle componenti, usando il framework per unit testing PHPUnit. 
\\
La procedura di unit testing e stata svolta nel seguente modo:
\begin{itemize}
    \item sono stati scritti dei brevi ma completi \textit{snippet} di codice che testino ogni funzionalità isolata, sia per quanto riguarda i casi di funzionamento corretto che quelli di funzionamento errato;
    
    \item sono stati raggruppati gli \textit{snippet} in TestCase;
    
    \item sono stati raggruppati i TestCase in TestSuite, che rappresentano una macro funzionalità rappresentata dal codice;
    
    \item infine è stato eseguito il codice di testing correggendo gli errori riscontrati ed assicurandosi che tutto il codice sia correttamente testato.
\end{itemize}

Sono stati inoltre eseguiti dei test funzionali, ovvero l'esecuzione dei casi d'uso previsti, simulando il comportamento atteso dall'utente, allo scopo di controllare che non si presentino bug o comportamenti imprevisti dell'applicazione.

\section{Validazione}
Lo scopo della validazione è accertare che il prodotto finale corrisponda alle attese, in modo da soddisfare tutti i requisiti prefissati inizialmente. I tipi di validazione effettuati sono stati due:
\begin{itemize}
    \item \textbf{validazione interna:} la validazione interna, chiamata anche "alfa test" o "pre-collaudo", è un processo che viene svolto da chi ha sviluppato il sistema. Al termine dello sviluppo del prodotto è stata effettuata la validazione interna in modo autonomo, simulando l'uso del prodotto da parte di un utente e verificando che tutte le funzionalità implementate funzionassero correttamente;
    
    \item \textbf{validazione esterna:} la validazione esterna, chiamata invece "beta-test" o "collaudo", è svolta dal committente del prodotto o dalla sua utenza. Il collaudo del prodotto sviluppato nel periodo di stage è stato fatto tramite una presentazione e dimostrazione dello stesso ai committenti.
\end{itemize}