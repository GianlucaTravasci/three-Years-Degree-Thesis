\chapter*{Glossario}
\label{cap:glossario}
%A

%B
\textbf{Backend}: Il backend è la parte di un software che elabora i dati generati dal frontend. Il backend incapsula la logica di elaborazione dei dati, e non interagisce direttamente con l’utilizzatore.
\\
\\
\textbf{Brainstorming}: L'espressione brainstorming, traducibile in lingua italiana come "assalto mentale" o "tempesta di cervelli", è una tecnica creativa di gruppo per far emergere idee volte alla risoluzione di un problema. Sinteticamente consiste, dato un problema, nell'organizzare una riunione in cui ogni partecipante propone liberamente soluzioni al problema, senza che nessuna di esse venga minimamente censurata. La critica ed eventuale selezione interverrà solo in un secondo tempo, terminata la seduta di brainstorming
\\
\\
\textbf{ERP}: significa Enterprise Resource Planning ("pianificazione delle risorse d'impresa"). Si tratta di un sistema di gestione che integra tutti i processi di business rilevanti di un'azienda (vendite, acquisti, gestione magazzino, contabilità ecc.).
\\
\\
%F
\textbf{Framework}: Un framework, in informatica e specificatamente nello sviluppo software, è un’architettura logica di supporto su cui un software può essere progettato e realizzato, spesso facilitandone lo sviluppo da parte del programmatore.
\\
\\
\textbf{Frontend}: Nel campo della progettazione software, il front-end è la parte di un sistema software che gestisce l’interazione con l’utente o con sistemi esterni che producono dati di ingresso.
\\
\\
%I
\textbf{IDE}: In informatica un ambiente di sviluppo integrato (in lingua inglese integrated development environment) è un software che, in fase di programmazione, aiuta i programmatori nello sviluppo del codice sorgente di un programma. Spesso l’IDE aiuta lo sviluppatore segnalando errori di sintassi del codice direttamente in fase di scrittura, oltre a tutta una serie di strumenti e funzionalità di supporto alla fase di sviluppo e debugging.
\\
\\
%J
\textbf{JSON}: (JavaScript Object Notation) Standard usato per trasmettere dati tramite lo
scambio di oggetti nei quali le informazioni sono salvate in coppie chiave-valore.
\\
\\
%M
\textbf{Metadato}:
informazioni aggiuntive che hanno attinenza con i documenti in fase di caricamento.
\\
\\
\textbf{MockFlow}:
MockFkow è un tool molto famoso per la sezione “Wireframe Pro“, ovvero un’applicazione concepita per la creazione di progetti di web design condivisibili con il team di lavoro. MockFlow è completamente interattivo e permette anche la realizzazione delle Sitemap delle pagine create. E’ possibile esportare in qualsiasi formato e fornisce una numerosa quantità di funzioni e strumenti.
\\
\\
%P
\textbf{PHP}: è un linguaggio di scripting interpretato, originariamente concepito per la programmazione di pagine web dinamiche.
\\
\\
\textbf{Plugin}: è un componente software che aggiunge specifiche funzionalità ad un programma esistente.
\\
\\
%R
\textbf{Refractoring}: in ingegneria del software, il refactoring (o code-refactoring) è una "tecnica strutturata per modificare la struttura interna di porzioni di codice senza modificarne il comportamento esterno", applicata per migliorare alcune caratteristiche non funzionali del software.
\\
\\
%S
\textbf{Software House}: azienda che si occupa dell'elaborazione e della commercializzazione di programmi per elaboratori.
\\
\\
\textbf{Software Intergration}: combinazione di subroutine, moduli software o programmi completi con altri componenti software per sviluppare un'applicazione o migliorare la funzionalità di uno esistente. Spesso richiedendo molte modifiche al codice sorgente, gli sviluppatori, così come il personale IT, possono dedicare gran parte del loro tempo a realizzare l'integrazione del software.
\\
\\
%U
\textbf{UML}: In ingegneria del software UML, Unified Modeling Language (ing. linguaggio di modellazione unificato) è un linguaggio di modellazione e specifica basato sul paradigma object-oriented. L’UML svolge un’importantissima funzione di “lingua franca” nella comunità della progettazione e programmazione a oggetti. Gran parte della letteratura di settore usa tale linguaggio per descrivere soluzioni analitiche e progettuali in modo sintetico e comprensibile a un vasto pubblico.
\\
\\
\textbf{XML}: è un metalinguaggio per la definizione di linguaggi di markup, ovvero un linguaggio marcatore basato su un meccanismo sintattico che consente di definire e controllare il significato degli elementi contenuti in un documento o in un testo.